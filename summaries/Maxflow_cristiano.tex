\documentclass[a4paper,10pt]{article}
\usepackage[utf8]{inputenc}
\usepackage{mathtools}
\usepackage[top=1in, bottom=1in, left=1in, right=1in]{geometry}
\newcommand{\wid}{\widetilde{O}}
\newcommand{\eps}{\epsilon}
%opening
\title{Electrical Flows, Laplacian Systems, and Faster Approximation of
Maximum Flow in Undirected Graphs \\ Cristiano, Kelner , et al}
\author{Karthik Abinav}
\date{\vspace{-5ex}}
\begin{document}
\maketitle

\section{Introduction}
  This paper gives a laplacian solver based algorithm to the max flow problem. Using electrical flows and fast laplacian solvers, the 
  running time to the (1-$\eps$)-approximate s-t flow problem in $\wid(mn^{\frac{1}{3}}\eps^{\frac{-11}{3}})$. The paper starts
  of by giving a simpler algorithm which runs in time $\wid(m^\frac{3}{2} \eps ^ \frac{5}{2})$. Then some key observation in this algorithm is used
  to get the modified algorithm that runs in time $\wid(m^\frac{4}{3} \eps^{-3})$. Then using the smoothing and sampling techniques given by 
  [insert bibtex link to karger's paper]. The main drawback with this algorithm is that, it can get the approximate value of the flow. But, 
  there is no technique to map the flow in the sparse graph to the original graph. The paper also discusses an algorithm to dual of this problem
  that is the minimum s-t cut. It gives a $(1+\eps)$- algorithm to the minimum s-t cut problem running in time $\wid(m+n^{\frac{4}{3}} \eps^{\frac{-8}{3}})$.
  The advantage of this algorithm is that, it can get the approximate cut along with the value of the minimum cut. Hence, a gap exists between
  the primal problem and the dual problem. This issue is addressed in a further work by Kelner, Orecchia, et. al.
  
 \section{Notations}
    In this section, some notations that will be used throughout this summary will be discussed. \\
    
    $u_e$ - This defines the capacity associated with the edge $e$. \\
    
    $U$ - This defines the ratio between the highest capacity to the lowest capacity in the graph i.e. $\frac{\displaystyle \max_{e} u_e}{\displaystyle \min_e u_e}$\\
    
    Every edge in the graph is arbitrarily oriented for notational purposes. An edge that comes into a vertex $v$ is denoted by $E^{+}(v)$ and
    every edge that leaves the vertex $v$ is defined as $E^{-}(v)$. \\
    
    A vector f is a vector of flow. The $i^{th}$ coordinate of the vector denotes the flow in the $i^{th}$ edge for a given ordering of the 
    edges \\
    
    Every valid flow of value F has the following properties: 
    \begin{itemize}
    \item 
      Flow Conservation: $$\sum_{\mathclap{e \in E^+(v)}} f(e)- \sum_{\mathclap{e \in E^-(v)}} f(e) = 0 ~~~~~ \forall v \in V \setminus \{s,t\}$$
    \item
      Net flow value:
      $$|f| = F = \sum_{\mathclap{e \in E^+(s)}} f(e)- \sum_{\mathclap{e \in E^-(s)}} f(e)$$
    \item
      Feasible flow :
      $$f(e) \leq u_e ~~~~~~ \forall e \in E$$
    \end{itemize}
    
    \textbf{The maximum flow problem}: The goal of this problem is to find a valid flow of value $F^{\ast}$ such that the maximum flow that 
    can be routed from s to t in the graph is $F^{\ast}$. \\
    A $(1-\eps)$-approximation is an algorithm that returns a valid s-t flow, whose value
    is atleast $(1-\eps)\ast F^{\ast}$ 
    
    \subsection{Electrical flows}
      Every edge of a graph G, we visualize an edge as a resistor with resistance $r_e$. We can then define an energy associated with a flow f as
      follows: 
      $$\eps_r(f) = \sum_{\mathclap{e}}r_e f_{e}^{2}$$

      \textbf{Lemma}: \\
	Among all valid flows of value F, the one that minimizes the energy is the one that follows kirchoff's laws of current and voltage \\
	
      \textbf{Proof}: \\
	Consider the case of two resistors in series and two resistors in parallel. If we prove the lemma for these two cases, it applies for 
	any general circuit. \\
	
	\textit{Series}: In this case, there is just one possible flow, which is the electrical flow. Hence, statement trivially holds true. \\
	
	\textit{Parallel}:  Let us start with an electrical flow. We will show that any change to the flows will only increase the energy.
	Let $i_1$ and $i_2$ be the electrical flow through resistors $r_1$ and $r_2$. From Kirchoff's law we have $i_1 \ast r_1 = i_2 \ast r_2 $.
	The energy now is $\eps = r_1  \ast i_1^2 + r_2 \ast i_2^2$. Let us now redistribute the flow as $i_1 + \delta$ and $i_2 - \delta$. 
	The value of the flow remains the same. The energy now becomes $\eps' = \eps + \delta^2(r_1+r_2) \geq \eps$. Hence the lemma. \\
	
	\textbf{Incidence Matrix}: The incidence matrix gives the details of the directed edges in a graph. The incidence matrix is denoted
	by $B$ a n $\ast$ m matrix. The (i,j) entry is 1, if the edge j leaves vertex i. It is defined as -1, if the edge j is incident 
	vertex i. \\
	
	Hence, the max flow problem can now be stated as calculating a vector f, such that.
	$$Bf = \chi_{s,t}$$
	Here, $\chi_{s,t}$ is a vector which is 1 at the $s^{th}$ coordinate, -1 at the $t^{th}$ coordinate and 0 elsewhere. \\
	
	\textbf{Resistance matrix}:
	   This is a diagonal matrix i.e. the off-diagonal entries are all 0. The (i,i) entry of the matrix contains the value $r_i$, that is
	   the resistance of the $i_th$ edge. \\
	   
	   This implies the energy can now be re-written as $R^TfR$ = $||R^{\frac{1}{2}}f||^2$ \\
	   
	\textbf{Effective Resistance}: For an electrical flow $f$ of value $1$ in a resistive electrical network, let $\phi$ be the vertex potentials.
	The effective resistance of the network with respect to the resistances $r$ is given by : 
	$$R_{eff}(r) = \phi(s) - \phi(t)$$
	
	\textbf{Graph Laplacian}: The weighted Laplacian of a graph $G$ with respect to edge resistances $r$ is defined as:
	$$BR^{-1}B^T$$
	
	\textbf{Congestion of an edge}: Congestion of an edge $e$ with respect to flow $f$ is defined as $cong_f(e)$ = $\frac{|f_e|}{u_e}$ \\
	
	\textbf{Lemma}: $R_{eff}(r) = \eps_r(f)$ \\
	    
	\textbf{Proof}: $R_{eff}(r) = \phi(s) - \phi(t) = \phi^T \chi_{s,t}$ \\
	
	Since $f$ is an electrical flow, f(u,v) = $\frac{\phi_u - \phi_v}{r_{u,v}}$. Hence, $f = R^{-1}B^T \phi$.\\
	
	$\eps_r(f) = f^TRf = (R^{-1}B^T\phi)^T R (R^{-1} B^T \phi) = \phi^T L \phi$ \\
	
	$L \phi = \chi_{s,t}$ \\
	
	$\eps_r(f) = (L^{-1}\chi_{s,t})^T L (L^{-1} \chi_{s,t}) = \chi_{s,t}^T L^{-1} \chi_{s,t}$ \\
	
	$R_{eff}(r) = (L^{-1}\chi_{s,t})^T \chi_{s,t}$
	
      \section{$\delta$-approximate flow}
	Given the following as input: \\
	$\delta>0$ \\
	$F>0$ \\
	vector $r$ of resistances in which the ratio of the largest to smallest resistance is atmost $R$ \\
	
	an algorithm returning the following output can be computed in time $\widetilde{O}(m \log(\frac{R}{\delta}))$ \\
	
	Vector potentials $\widetilde{\phi}$, An s-t flow $\widetilde{f}$ of value F. Additionally the output satisfies the following : 
	\begin{itemize}
	 \item 
	  $\eps_r(\widetilde{f}) \leq \eps_r(f)\ast [1+\delta]$
	 \item
	  For every edge e, \\
	  
	  $|r_ef_e^2 - r_e\widetilde{f}^2| \leq \frac{\delta}{2mR}\eps_r(f)$
	 \item
	  $\widetilde{\phi}(s)-\widetilde{\phi}(t) \geq (1-\frac{\delta}{12nmR})\ast F \ast R_{eff}(r)$
	\end{itemize}

	The algorithms that follow will use the above algorithm as a sub-routine to obtain a $\delta$-approximate electrical flow, for appropriate
	values of $\delta$.
	
	\section{A $\widetilde{O}(m^{\frac{3}{2}}\eps^{\frac{-5}{2}})$ time flow algorithm}
	  In this section, first we will have a look at a simpler version of the algorithm. After taking key insights from this algorithm,
	  we can make appropriate observations and modifications to improve the running time. \\
	  
	  \subsection{$(\eps,\rho)$- oracle}
	    This is the second sub-routine that will be used by the algorithm. The construction of such an algorithm uses the $\delta$-approximate
	    electrical flow algorithm. \\
	    
	    
	    \textbf{Input} \\
	    Real number $F>0$ \\
	    $w$ - vector of edge weights, where $w_e \geq 1 ~~ \forall e \in E$ \\
	    
	    \textbf{Output}\\
	    A s-t flow $f$ satisfying the following criteria \\
	    
	    \begin{itemize}
	     \item 
	      If $F \leq F^{\ast}$, output a flow f s.t.
		\begin{itemize}
		 \item 
		    $|f| = F$
		 \item
		    $\frac{\sum\limits_{e} w_e \ast cong_f(e)}{\sum\limits_{e} w_e} \leq (1+\eps)$
		 \item
		    $\max\limits_{e} cong_f(e) \leq \rho$
		\end{itemize}

	     \item
		If $F > F^{\ast}$, output a flow $f$ sastisfying the above or ``fail''.
	  \end{itemize}

	  \subsection{Main Algorithm}
	      The main algorithm calls the $(\eps,\rho)$ subroutine many times with different weights and then takes an average of all the values
	      returned by the oracle as the required value of flow. Note, the flow returned by this algorithm should maintain the congestion criteria
	      for each of the edges. On the other hand, the oracle may return a flow whose congestion at each edge is as bad as $\rho$. But, since
	      the oracle gives a upper bound on the average congestion, the algorithm uses that critically to, in some sense ``correct'' the value of 
	      flows. \\
	      
	      \textbf{Input} \\
	      A graph $G$ \\
	      a vector u of edge capacities \\
	      A target flow value F \\
	      ($\eps, \rho)$ - oracle O \\
	      
	      \textbf{Output}\\
	      Either a flow $f$ or ``fail'' indicating $F > F^{\ast}$ \\
	      
	      Note, this algorithm requires a target flow value $F$ as an input. Hence, to get a value of maximum flow, we can perform a recursive
	      doubling and binary search on the value of $F$ to get to the value $F^{\ast}$. \\
	      
	      The algorithm goes as follows: \\
	      
	      \textbf{Note}: The superscript on the weight vector impies the weights in the $i^{th}$ iteration of the algorithm.
	      \begin{itemize}
	       \item 
		Initialize the weights $w^0$ as $w_e =1$ for all the edges e. Initialize N as $\frac{2\rho \ln(m)}{\eps^2}$
	      \item
		for i from 1 to N:
		  \begin{itemize}
		   \item 
		      Query the oracle O with $w^{i-1}$ and target flow as $F$
		   \item
		      if O returns a ``fail'', return ``fail''
		   \item
		      else: \\
		      	Let $f^i$ be the value of the returned flow. Update weights as
			 $$w^{i} = w^{i-1}\ast\left(1+\frac{\eps\ast cong_{f^i}(e)}{\rho}\right)$$
		  \end{itemize}

		\item
		  Return $f$ as $\frac{(1-\eps)^2}{(1+\eps)N}\left(\sum\limits_{i} f^i\right)$
	      \end{itemize}
	
	  \section{Analysis of the simple algorithm}
	    \subsection{Construction of an ($\eps$, $3\sqrt{\frac{m}{\eps}}$)-oracle}
		Let us now see, how to construct an ($\eps$,$\rho$)-oracle with parameter $\rho = 3\sqrt{\frac{m}{\eps}}$. \\
		
		\textbf{Input} \\
	      A graph $G$ \\
	      A vector u of edge capacities \\
	      A target flow value F \\
	      Edge weights vector $w$
	      
	      \textbf{Output}\\
	      Either a flow $f$ or ``fail'' indicating $F > F^{\ast}$ \\
	      
	      \begin{itemize}
	       \item 
		  Set $r_e$ = $\frac{1}{u_e^2}\left(w_e + \frac{\eps}{3m}|w|_1\right)$, for every edge $e \in E$
		\item	
		  Now call the $\delta$-approximate electrical flow construction with the above resistances, given graph, the target flow value F
		  and the value of $\delta$ as $\frac{\eps}{3}$.
		\item
		  Let $\widetilde{f}$ be the returned flow. If $\eps_r(\widetilde{f}) > (1+\eps)|w|_1$, then return fail.
		  Else, return $\widetilde{f}$.
	      \end{itemize}

	\subsection{Analysis of the construction}
	  We have to now show that, the above construction indeed satsifies the properties of an ($\eps$, $\rho$)-oracle. \\
	  
	  1) It is immediately clear that $|\widetilde{f}| = F$. This is because the approximate electrical flow construction returns a flow with
	  a target flow value of F. \\
	  
	  2) We need to show that, whenever $F \leq F^{\ast}$ the construction doesn't return a fail. To show this consider the following: \\
	  
	    Let $f^{\ast}$ be a vector corresponding to the exact maximum flow in the graph. Clearly, $cong_{f^{\ast}}(e) \leq 1$ for all the edges $e \in E$.
	    $$\eps_r(f^{\ast}) = \sum\limits_e \left(w_e + \frac{\eps}{3m}|w|_1\right)\left( \frac{f^{\ast}}{u_e} \right)^2$$
	    $$= \sum\limits_e \left(w_e + \frac{\eps}{3m}|w|_1\right)\left( cong_{f^{\ast}}(e) \right)^2$$
	    $$\leq \sum\limits_e \left(w_e + \frac{\eps}{3m}|w|_1\right)$$
	    $$\leq \sum\limits_e w_e + \frac{\eps}{3}|w|_1$$
	    $$\leq |w|_1 \left(1 + \frac{\eps}{3} \right)$$
	  
	  Let $f$ be the electrical flow of value $F$. 
	      

\end{document}
